\chapter{Abstract Algebra}

Abstract algebra is the study of algebraic structures defined by axiomatic systems. Unlike elementary algebra, which focuses on solving equations involving real or complex numbers, abstract algebra generalizes these concepts to analyze structures that obey specific algebraic laws. It abstracts the common properties of diverse mathematical systems—such as integers, symmetry transformations, matrices, and polynomials—allowing us to reason about them in a unified framework. 

This chapter provides a rigorous exploration of four pillars of algebra: \textbf{Groups}, \textbf{Rings}, \textbf{Fields}, and \textbf{Modules}. We emphasize axiomatic definitions, structural theorems, and the interplay between these systems.

\section{Groups}

Groups are the fundamental structures for studying symmetry. A group abstracts the notion of invertible operations, whether they are geometric rotations, permutations of a set, or arithmetic addition.

\subsection{Definition and Examples}

\begin{definition}[Group]
A \textbf{group} is a set $G$ equipped with a binary operation $\cdot: G \times G \to G$ satisfying the following axioms:
\begin{enumerate}
    \item \textbf{Associativity}: For all $a, b, c \in G$, $(a \cdot b) \cdot c = a \cdot (b \cdot c)$.
    \item \textbf{Identity Element}: There exists a unique element $e \in G$ (often denoted $1$ or $0$ depending on context) such that for all $a \in G$, $a \cdot e = e \cdot a = a$.
    \item \textbf{Inverses}: For every $a \in G$, there exists a unique element $a^{-1} \in G$ such that $a \cdot a^{-1} = a^{-1} \cdot a = e$.
\end{enumerate}
(Note: The closure property is implicit in the definition of a binary operation $G \times G \to G$).
\end{definition}

\begin{definition}[Abelian Group]
If the operation is commutative (i.e., $a \cdot b = b \cdot a$ for all $a, b \in G$), the group is called \textbf{abelian}.
\end{definition}

\newpage

\begin{example}[Fundamental Examples]
\begin{enumerate}
    \item \textbf{Integers}: $(Z, +)$ is an infinite abelian group with identity $0$ and inverse $-a$.
    \item \textbf{General Linear Group}: The set $GL_n(R)$ of invertible $n \times n$ matrices with real entries is a non-abelian group under matrix multiplication.
    \item \textbf{Symmetric Group}: $S_n$, the set of all bijections from $\{1, \dots, n\}$ to itself, is a group under composition. $|S_n| = n!$. It is non-abelian for $n \geq 3$.
    \item \textbf{Cyclic Groups}: $Z_n$ (integers modulo $n$) under addition is a cyclic group of order $n$.
    \item \textbf{Dihedral Groups}: $D_{2n}$ represents the symmetries of a regular $n$-gon, containing $n$ rotations and $n$ reflections. $|D_{2n}| = 2n$.
\end{enumerate}
\end{example}

\subsection{Elementary Properties}

The axioms imply strong structural regularities.

\begin{proposition}[Cancellation Laws]
Let $G$ be a group and $a, b, c \in G$.
\begin{enumerate}
    \item If $ab = ac$, then $b=c$ (Left Cancellation).
    \item If $ba = ca$, then $b=c$ (Right Cancellation).
    \item $(ab)^{-1} = b^{-1}a^{-1}$ (The "Shoe-Sock" Property).
\end{enumerate}
\end{proposition}

\begin{definition}[Order]
The \textbf{order of a group} $G$, denoted $|G|$, is the cardinality of the set $G$. The \textbf{order of an element} $g \in G$, denoted $|g|$, is the smallest positive integer $n$ such that $g^n = e$. If no such $n$ exists, $g$ has infinite order.
\end{definition}

\subsection{Subgroups and Cosets}

\begin{definition}[Subgroup]
A subset $H \subseteq G$ is a \textbf{subgroup} (denoted $H \leq G$) if $H$ is a group under the restricted operation of $G$.
\end{definition}

\begin{lemma}[Subgroup Test]
A non-empty subset $H \subseteq G$ is a subgroup if and only if for all $x, y \in H$, $xy^{-1} \in H$.
\end{lemma}

\begin{definition}[Cosets]
Let $H \leq G$. For any $g \in G$:
\begin{itemize}
    \item The \textbf{left coset} of $H$ containing $g$ is $gH = \{gh \mid h \in H\}$.
    \item The \textbf{right coset} of $H$ containing $g$ is $Hg = \{hg \mid h \in H\}$.
\end{itemize}
\end{definition}

Cosets partition the group $G$. Importantly, all cosets of a subgroup $H$ have the same cardinality as $H$. This leads to one of the most famous theorems in finite group theory.

\begin{theorem}[Lagrange's Theorem]
If $G$ is a finite group and $H \leq G$, then $|H|$ divides $|G|$. Furthermore,
\[ |G| = [G:H] \cdot |H|, \]
where $[G:H]$ is the number of distinct left cosets of $H$ in $G$, called the \textbf{index}.
\end{theorem}

\begin{corollary}
    If $|G| = p$ where $p$ is a prime number, then $G$ is cyclic and essentially unique (isomorphic to $Z_p$).
\end{corollary}

\subsection{Normal Subgroups and Quotient Groups}

Not all subgroups are created equal. To construct a new group from cosets, we require the subgroup to be "normal."

\begin{definition}[Normal Subgroup]
A subgroup $N \leq G$ is \textbf{normal}, denoted $N \trianglelefteq G$, if it is invariant under conjugation. That is, for all $g \in G$ and $n \in N$, $gng^{-1} \in N$.
\end{definition}

\begin{proposition}
The following are equivalent:
\begin{enumerate}
    \item $N \trianglelefteq G$.
    \item $gN = Ng$ for all $g \in G$ (Left cosets equal right cosets).
    \item The operation $(aN)(bN) := (ab)N$ is well-defined.
\end{enumerate}
\end{proposition}

\begin{definition}[Quotient Group]
If $N \trianglelefteq G$, the set of cosets $G/N$ forms a group under the operation defined above. This is called the \textbf{quotient group}. The order is $|G/N| = [G:N]$.
\end{definition}

\subsection{Homomorphisms and Isomorphisms}

\begin{definition}[Homomorphism]
A function $\phi: G \to H$ is a \textbf{homomorphism} if $\phi(xy) = \phi(x)\phi(y)$ for all $x, y \in G$.
\end{definition}

Associated with any homomorphism are two structural components:
\begin{itemize}
    \item \textbf{Kernel}: $Ker(\phi) = \{g \in G \mid \phi(g) = e_H\}$. This is always a \textit{normal} subgroup of $G$.
    \item \textbf{Image}: $Img(\phi) = \{\phi(g) \mid g \in G\}$. This is a subgroup of $H$.
\end{itemize}

\begin{theorem}[First Isomorphism Theorem]
Let $\phi: G \to H$ be a homomorphism. Then there is an isomorphism:
\[ G / Ker(\phi) \cong Img(\phi). \]
\end{theorem}
This theorem essentially states that the image of a group looks exactly like the group "modulo" the elements that are sent to identity.

\begin{theorem}[Second and Third Isomorphism Theorems]
\begin{enumerate}
    \item Let $H \leq G$ and $N \trianglelefteq G$. Then $H \cap N \trianglelefteq H$ and $H/(H \cap N) \cong HN/N$.
    \item Let $N \trianglelefteq G$ and $K \trianglelefteq G$ with $N \leq K$. Then $(G/N)/(K/N) \cong G/K$.
\end{enumerate}
\end{theorem}

\subsection{Group Actions and Sylow Theorems}

Group actions provide a dynamic view of groups as "doers" rather than just static structures.

\begin{definition}[Group Action]
A group $G$ \textbf{acts} on a set $X$ if there is a map $G \times X \to X$, denoted $g \cdot x$, such that $e \cdot x = x$ and $g \cdot (h \cdot x) = (gh) \cdot x$.
\end{definition}

Key concepts include the \textbf{Orbit} $Orb(x) = \{g \cdot x \mid g \in G\}$ and the \textbf{Stabilizer} $Stab(x) = \{g \in G \mid g \cdot x = x\}$.

\begin{theorem}[Orbit-Stabilizer]
For a finite group $G$ acting on $X$, $|Orb(x)| = |G| / |Stab(x)|$.
\end{theorem}

\begin{theorem}[The Class Equation]
Let $G$ act on itself by conjugation ($g \cdot x = gxg^{-1}$). The orbits are called conjugacy classes. We have:
\[ |G| = |Z(G)| + \sum_{i} [G : C_G(g_i)], \]
where $Z(G)$ is the center of $G$, and the sum runs over representatives of distinct non-central conjugacy classes.
\end{theorem}

\begin{theorem}[Sylow Theorems]
Let $|G| = p^k m$ with $p \nmid m$.
\begin{enumerate}
    \item \textbf{Existence}: $G$ has a subgroup of order $p^k$ (Sylow $p$-subgroup).
    \item \textbf{Conjugacy}: All Sylow $p$-subgroups are conjugate.
    \item \textbf{Number}: Let $n_p$ be the number of Sylow $p$-subgroups. Then $n_p \equiv 1 \pmod p$ and $n_p \mid m$.
\end{enumerate}
\end{theorem}

\section{Rings}

Rings are sets equipped with two binary operations, usually modeling "arithmetic" where we can add, subtract, and multiply, but not necessarily divide.

\subsection{Fundamentals}

\begin{definition}[Ring]
A \textbf{ring} $R$ is a set with operations $(+, \cdot)$ such that:
\begin{enumerate}
    \item $(R, +)$ is an abelian group (identity $0$).
    \item $(R, \cdot)$ is associative.
    \item The Distributive Laws hold: $a(b+c) = ab + ac$ and $(a+b)c = ac + bc$.
\end{enumerate}
If there is a multiplicative identity $1 \neq 0$, $R$ is a \textbf{ring with unity}. If $ab=ba$, $R$ is \textbf{commutative}.
\end{definition}

\begin{definition}[Types of Elements]
\begin{itemize}
    \item \textbf{Unit}: An element $u$ is a unit if it has a multiplicative inverse.
    \item \textbf{Zero Divisor}: A non-zero element $a$ is a zero divisor if $\exists b \neq 0$ such that $ab=0$.
    \item \textbf{Integral Domain}: A commutative ring with unity and no zero divisors.
\end{itemize}
\end{definition}

\subsection{Ideals and Homomorphisms}

Ideals are to rings what normal subgroups are to groups: they allow the construction of quotients.

\begin{definition}[Ideal]
A subset $I \subseteq R$ is a (two-sided) \textbf{ideal} if:
\begin{enumerate}
    \item $(I, +)$ is a subgroup of $(R, +)$.
    \item Absorbency: For all $r \in R$ and $x \in I$, both $rx \in I$ and $xr \in I$.
\end{enumerate}
\end{definition}

\begin{definition}[Prime and Maximal Ideals]
Let $R$ be a commutative ring with unity.
\begin{itemize}
    \item An ideal $P \subsetneq R$ is \textbf{prime} if $ab \in P \implies a \in P$ or $b \in P$.
    \item An ideal $M \subsetneq R$ is \textbf{maximal} if there is no ideal $I$ such that $M \subsetneq I \subsetneq R$.
\end{itemize}
\end{definition}

\begin{theorem}[Quotients by Special Ideals]
\begin{enumerate}
    \item $R/P$ is an Integral Domain $\iff P$ is a prime ideal.
    \item $R/M$ is a Field $\iff M$ is a maximal ideal.
\end{enumerate}
\end{theorem}

\subsection{Polynomial Rings and Divisibility}

Let $R$ be an integral domain.
\begin{itemize}
    \item \textbf{Euclidean Domain (ED)}: A domain with a division algorithm (e.g., $Z, F[x]$).
    \item \textbf{Principal Ideal Domain (PID)}: A domain where every ideal is generated by one element ($I = \langle a \rangle$).
    \item \textbf{Unique Factorization Domain (UFD)}: A domain where every non-zero non-unit factors uniquely into irreducibles.
\end{itemize}

\begin{theorem}[Hierarchy of Domains]
\[ \text{Fields} \subset \text{Euclidean Domains} \subset \text{PIDs} \subset \text{UFDs} \subset \text{Integral Domains} \]
\end{theorem}

\begin{theorem}[Gauss's Lemma]
If $R$ is a UFD, then the polynomial ring $R[x]$ is also a UFD. Consequently, $Z[x]$ is a UFD, even though it is not a PID.
\end{theorem}

\section{Fields}

Fields are commutative rings where division (by non-zero elements) is always defined. They are the setting for linear algebra and Galois theory.

\subsection{Extensions}

\begin{definition}
If $F \subseteq K$ are fields, $K$ is an \textbf{extension} of $F$, denoted $K/F$. The \textbf{degree} $[K:F]$ is the dimension of $K$ as an $F$-vector space.
\end{definition}

\begin{theorem}[Tower Law]
If $F \subseteq L \subseteq K$, then $[K:F] = [K:L][L:F]$.
\end{theorem}

\begin{definition}
Let $\alpha \in K$.
\begin{itemize}
    \item $\alpha$ is \textbf{algebraic} over $F$ if it is the root of some polynomial $f(x) \in F[x]$.
    \item The \textbf{minimal polynomial} of $\alpha$ is the unique monic irreducible polynomial in $F[x]$ having $\alpha$ as a root.
\end{itemize}
If all elements of $K$ are algebraic over $F$, $K/F$ is an \textbf{algebraic extension}.
\end{definition}

\subsection{Splitting Fields and Algebraic Closure}

\begin{definition}[Splitting Field]
The splitting field of a polynomial $f(x) \in F[x]$ is the smallest extension $K/F$ in which $f(x)$ decomposes into linear factors $(x-\alpha_1)\dots(x-\alpha_n)$.
\end{definition}

\begin{definition}[Algebraic Closure]
A field $\bar{F}$ is algebraically closed if every non-constant polynomial in $\bar{F}[x]$ has a root in $\bar{F}$. Every field $F$ has a unique (up to isomorphism) algebraic closure.
\end{definition}

\newpage

\subsection{Finite Fields}

Finite fields are fully classified.
\begin{theorem}
Let $F$ be a finite field.
\begin{enumerate}
    \item The characteristic of $F$ is a prime $p$.
    \item The number of elements is $|F| = p^n$ for some $n \geq 1$.
    \item For every prime $p$ and integer $n$, there is a unique finite field of order $p^n$, denoted $F_{p^n}$ or $GF(p^n)$.
    \item $F_{p^n}$ is the splitting field of $x^{p^n} - x$ over $F_p$.
\end{enumerate}
\end{theorem}

\section{Galois Theory}

Galois Theory relates field extensions to groups of automorphisms, solving ancient problems like the impossibility of trisecting an angle or solving quintic equations by radicals.

\subsection{The Galois Correspondence}

\begin{definition}
Let $K/F$ be an extension. The \textbf{Galois Group}, $Gal(K/F)$, is the set of all automorphisms $\sigma: K \to K$ such that $\sigma(a) = a$ for all $a \in F$.
\end{definition}

\begin{definition}[Galois Extension]
An extension $K/F$ is \textbf{Galois} if it is:
\begin{enumerate}
    \item \textbf{Normal}: Irreducible polynomials in $F[x]$ with a root in $K$ split completely in $K$.
    \item \textbf{Separable}: Irreducible polynomials over $F$ have distinct roots in algebraic closure (no multiple roots).
\end{enumerate}
\end{definition}

\begin{theorem}[Fundamental Theorem of Galois Theory]
Let $K/F$ be a finite Galois extension with Galois group $G = Gal(K/F)$. There is a one-to-one inclusion-reversing correspondence between subgroups $H \leq G$ and intermediate fields $F \subseteq E \subseteq K$.
Specifically:
\begin{enumerate}
    \item The fixed field of $H$ is $E$.
    \item $E$ is a normal extension of $F$ if and only if $H$ is a normal subgroup of $G$. In this case, $Gal(E/F) \cong G/H$.
\end{enumerate}
\end{theorem}

\subsection{Solvability by Radicals}

\begin{definition}
A group $G$ is \textbf{solvable} if there is a chain $1 = G_0 \trianglelefteq G_1 \trianglelefteq \dots \trianglelefteq G_n = G$ where $G_{i+1}/G_i$ is abelian.
\end{definition}

\begin{theorem}
A polynomial $f(x)$ is solvable by radicals (using $n$-th roots) if and only if its Galois group is a solvable group. Since $S_n$ is not solvable for $n \geq 5$, there is no general quintic formula.
\end{theorem}

\section{Modules}

Modules are generalizations of vector spaces where the scalars come from a ring $R$ rather than a field. This seemingly small change adds significant complexity (e.g., lack of bases).

\subsection{Definitions}

\begin{definition}[R-Module]
Let $R$ be a ring. A left \textbf{$R$-module} $M$ is an abelian group $(M, +)$ equipped with an action $R \times M \to M$ such that for all $r, s \in R$ and $m, n \in M$:
\begin{enumerate}
    \item $r(m+n) = rm + rn$.
    \item $(r+s)m = rm + sm$.
    \item $(rs)m = r(sm)$.
    \item $1m = m$ (if $R$ has unity).
\end{enumerate}
\end{definition}

\begin{example}
\begin{itemize}
    \item Any vector space over $F$ is an $F$-module.
    \item Any abelian group $G$ is a $Z$-module ($n \cdot g$ is repeated addition).
    \item $R$ itself is an $R$-module.
    \item An ideal $I \subseteq R$ is an $R$-submodule of $R$.
\end{itemize}
\end{example}

\subsection{Module Homomorphisms and Exact Sequences}

\begin{definition}
An $R$-module homomorphism is a map $f: M \to N$ respecting addition and scalar multiplication.
\end{definition}

\begin{definition}[Exact Sequence]
A sequence of modules and homomorphisms
\[ \dots \xrightarrow{f_{i-1}} M_{i} \xrightarrow{f_i} M_{i+1} \xrightarrow{f_{i+1}} \dots \]
is \textbf{exact} at $M_i$ if $Img(f_{i-1}) = Ker(f_i)$.
\end{definition}

A \textbf{Short Exact Sequence} $0 \to A \xrightarrow{f} B \xrightarrow{g} C \to 0$ implies $A$ embeds into $B$ and $C \cong B/A$.

\subsection{Finitely Generated Modules over PIDs}

This is the crowning theorem of basic module theory, generalizing the Fundamental Theorem of Finite Abelian Groups and the Jordan Canonical Form.

\begin{theorem}[Structure Theorem]
Let $R$ be a PID and $M$ a finitely generated $R$-module. Then $M$ decomposes uniquely as:
\[ M \cong R^k \oplus R/\langle d_1 \rangle \oplus R/\langle d_2 \rangle \oplus \dots \oplus R/\langle d_m \rangle \]
where $k \geq 0$ is the \textbf{rank}, and $d_1 \mid d_2 \mid \dots \mid d_m$ are non-zero non-units called the \textbf{invariant factors}.
\end{theorem}