\preface
\thispagestyle{empty}

At the heart of mathematics lies a beautiful and fundamental tension: the tension between our innate, intuitive grasp of the world and the uncompromising demand for absolute certainty.

We all begin as naive mathematicians. We perceive patterns, sense relationships, and manipulate numbers and shapes with an instinctive confidence. This "naive understanding" is the soil from which all mathematical curiosity grows. It is natural, powerful, and profoundly human.

Yet, as history has shown, intuition alone can be a treacherous guide, leading to contradictions and uncertainties when pushed beyond its limits. The great edifice of modern mathematics, therefore, could not be built upon this soil alone. It required foundations dug deep into the bedrock of logical rigor.

For hundreds of generations, starting from the simplest details, mathematicians have continuously abstracted mathematical concepts and built an edifice of logic. Geometers began with the most basic geometric structures—points, lines, and planes—establishing the system of Euclidean plane geometry through careful postulates and proofs. This foundational framework, with its emphasis on congruence, similarity, and the properties of space, eventually developed and expanded into modern advanced geometry, where non-Euclidean alternatives challenged long-held assumptions about parallel lines and curvature. From there, it blossomed into topology, the study of shapes and spaces that remain invariant under continuous deformations, and even gave rise to concepts such as manifolds, which provide the mathematical scaffolding for understanding higher-dimensional realities in physics and beyond.

Algebraists, meanwhile, started from the most fundamental concept—quantity itself—and built simple, elementary algebra around operations like addition, subtraction, and solving equations for unknowns. Over time, they further abstracted algebraic structures, recognizing patterns in groups, rings, and fields, leading to disciplines like linear algebra, which models vector spaces and transformations essential to everything from computer graphics to quantum mechanics, and abstract algebra as we know it today, a realm of pure structure that underpins cryptography, coding theory, and the symmetries of the universe.

This process of abstraction extended to other branches as well. In analysis, scholars began with the intuitive notions of limits and continuity, formalizing them into the rigorous calculus of Newton and Leibniz, which evolved into real and complex analysis, measure theory, and functional analysis—tools that allow us to grapple with infinities, probabilities, and the behavior of functions in infinite-dimensional spaces. Number theorists, drawing from the primal fascination with integers and primes, constructed arithmetic systems that branched into analytic number theory, algebraic number theory, and even the profound mysteries of the Riemann Hypothesis, connecting primes to the zeros of complex functions.

This book is about the construction of that edifice. It is the story of the journey from the fertile but fuzzy landscape of intuition to the crystalline structure of formal axiomatic systems. We will witness how mathematicians, through centuries of intellectual toil, learned to distill their intuitive ideas into precise definitions and unequivocal rules—axioms.

These axioms are the cornerstone of the mathematical skyscraper. Chosen with care, they are both simple enough to be self-evident and powerful enough to support an ever-ascending tower of ideas. Each new floor—a theorem, a theory, a whole new discipline like calculus or algebra—is constructed securely upon the layers beneath it, its integrity guaranteed by the logical connections that bind it to the foundation.

This architectural principle is what makes mathematics the most unifying of all languages. It connects the geometric world of shapes with the algebraic world of equations, and the discrete world of integers with the continuous world of analysis, weaving them into a single, grand, and coherent narrative. It bridges the probabilistic uncertainties of statistics with the deterministic certainties of logic, and even links the abstract realms of set theory—where infinities are tamed and paradoxes resolved—with the applied worlds of computer science and engineering.

The skyscraper stands as a testament to the power of human reason. Yet, Gödel's incompleteness theorems cast a fascinating and necessary shadow. They remind us that even the most perfectly constructed skyscraper cannot contain the tools to verify the stability of its own deepest foundations from within. This is not a flaw that collapses the structure; rather, it is a profound insight into its very nature.

It tells us that mathematics is not a static, completed temple of absolute truth, but a living, growing, and endlessly fascinating human adventure. The inability to achieve a final, self-verifying system is not a weakness, but a source of strength—it guarantees that the adventure will never end, that there will always be new horizons to explore and new questions to ask. It invites us to embrace the unknown, to push the boundaries of what we can formalize, and to find beauty in the interplay between certainty and mystery.

This manuscript is an invitation to join this great adventure. It is designed for those who are not satisfied with merely being told a result; it is for those who wish to stand at the drawing board and understand, step by logical step, how the skyscraper was designed and built. Along the way, we will encounter the triumphs of discovery, the pitfalls of early misconceptions, and the elegant resolutions that have shaped the field into what it is today.

With each new concept mastered, the view from the skyscraper grows more magnificent.

Welcome.

\begin{flushright}
Jinshuo Li \\
Shanghai Jiao Tong University \\
2025.12.2, in Shanghai
\end{flushright}

\cleardoublepage
\pagenumbering{arabic}
\setcounter{page}{1}